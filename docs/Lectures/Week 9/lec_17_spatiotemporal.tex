% Options for packages loaded elsewhere
\PassOptionsToPackage{unicode}{hyperref}
\PassOptionsToPackage{hyphens}{url}
%
\documentclass[
  ignorenonframetext,
]{beamer}
\usepackage{pgfpages}
\setbeamertemplate{caption}[numbered]
\setbeamertemplate{caption label separator}{: }
\setbeamercolor{caption name}{fg=normal text.fg}
\beamertemplatenavigationsymbolsempty
% Prevent slide breaks in the middle of a paragraph
\widowpenalties 1 10000
\raggedbottom
\setbeamertemplate{part page}{
  \centering
  \begin{beamercolorbox}[sep=16pt,center]{part title}
    \usebeamerfont{part title}\insertpart\par
  \end{beamercolorbox}
}
\setbeamertemplate{section page}{
  \centering
  \begin{beamercolorbox}[sep=12pt,center]{part title}
    \usebeamerfont{section title}\insertsection\par
  \end{beamercolorbox}
}
\setbeamertemplate{subsection page}{
  \centering
  \begin{beamercolorbox}[sep=8pt,center]{part title}
    \usebeamerfont{subsection title}\insertsubsection\par
  \end{beamercolorbox}
}
\AtBeginPart{
  \frame{\partpage}
}
\AtBeginSection{
  \ifbibliography
  \else
    \frame{\sectionpage}
  \fi
}
\AtBeginSubsection{
  \frame{\subsectionpage}
}
\usepackage{lmodern}
\usepackage{amssymb,amsmath}
\usepackage{ifxetex,ifluatex}
\ifnum 0\ifxetex 1\fi\ifluatex 1\fi=0 % if pdftex
  \usepackage[T1]{fontenc}
  \usepackage[utf8]{inputenc}
  \usepackage{textcomp} % provide euro and other symbols
\else % if luatex or xetex
  \usepackage{unicode-math}
  \defaultfontfeatures{Scale=MatchLowercase}
  \defaultfontfeatures[\rmfamily]{Ligatures=TeX,Scale=1}
\fi
% Use upquote if available, for straight quotes in verbatim environments
\IfFileExists{upquote.sty}{\usepackage{upquote}}{}
\IfFileExists{microtype.sty}{% use microtype if available
  \usepackage[]{microtype}
  \UseMicrotypeSet[protrusion]{basicmath} % disable protrusion for tt fonts
}{}
\makeatletter
\@ifundefined{KOMAClassName}{% if non-KOMA class
  \IfFileExists{parskip.sty}{%
    \usepackage{parskip}
  }{% else
    \setlength{\parindent}{0pt}
    \setlength{\parskip}{6pt plus 2pt minus 1pt}}
}{% if KOMA class
  \KOMAoptions{parskip=half}}
\makeatother
\usepackage{xcolor}
\IfFileExists{xurl.sty}{\usepackage{xurl}}{} % add URL line breaks if available
\IfFileExists{bookmark.sty}{\usepackage{bookmark}}{\usepackage{hyperref}}
\hypersetup{
  pdftitle={Fitting spatial and spatiotemporal models},
  pdfauthor={Eric Ward},
  hidelinks,
  pdfcreator={LaTeX via pandoc}}
\urlstyle{same} % disable monospaced font for URLs
\newif\ifbibliography
\usepackage{color}
\usepackage{fancyvrb}
\newcommand{\VerbBar}{|}
\newcommand{\VERB}{\Verb[commandchars=\\\{\}]}
\DefineVerbatimEnvironment{Highlighting}{Verbatim}{commandchars=\\\{\}}
% Add ',fontsize=\small' for more characters per line
\usepackage{framed}
\definecolor{shadecolor}{RGB}{248,248,248}
\newenvironment{Shaded}{\begin{snugshade}}{\end{snugshade}}
\newcommand{\AlertTok}[1]{\textcolor[rgb]{0.94,0.16,0.16}{#1}}
\newcommand{\AnnotationTok}[1]{\textcolor[rgb]{0.56,0.35,0.01}{\textbf{\textit{#1}}}}
\newcommand{\AttributeTok}[1]{\textcolor[rgb]{0.77,0.63,0.00}{#1}}
\newcommand{\BaseNTok}[1]{\textcolor[rgb]{0.00,0.00,0.81}{#1}}
\newcommand{\BuiltInTok}[1]{#1}
\newcommand{\CharTok}[1]{\textcolor[rgb]{0.31,0.60,0.02}{#1}}
\newcommand{\CommentTok}[1]{\textcolor[rgb]{0.56,0.35,0.01}{\textit{#1}}}
\newcommand{\CommentVarTok}[1]{\textcolor[rgb]{0.56,0.35,0.01}{\textbf{\textit{#1}}}}
\newcommand{\ConstantTok}[1]{\textcolor[rgb]{0.00,0.00,0.00}{#1}}
\newcommand{\ControlFlowTok}[1]{\textcolor[rgb]{0.13,0.29,0.53}{\textbf{#1}}}
\newcommand{\DataTypeTok}[1]{\textcolor[rgb]{0.13,0.29,0.53}{#1}}
\newcommand{\DecValTok}[1]{\textcolor[rgb]{0.00,0.00,0.81}{#1}}
\newcommand{\DocumentationTok}[1]{\textcolor[rgb]{0.56,0.35,0.01}{\textbf{\textit{#1}}}}
\newcommand{\ErrorTok}[1]{\textcolor[rgb]{0.64,0.00,0.00}{\textbf{#1}}}
\newcommand{\ExtensionTok}[1]{#1}
\newcommand{\FloatTok}[1]{\textcolor[rgb]{0.00,0.00,0.81}{#1}}
\newcommand{\FunctionTok}[1]{\textcolor[rgb]{0.00,0.00,0.00}{#1}}
\newcommand{\ImportTok}[1]{#1}
\newcommand{\InformationTok}[1]{\textcolor[rgb]{0.56,0.35,0.01}{\textbf{\textit{#1}}}}
\newcommand{\KeywordTok}[1]{\textcolor[rgb]{0.13,0.29,0.53}{\textbf{#1}}}
\newcommand{\NormalTok}[1]{#1}
\newcommand{\OperatorTok}[1]{\textcolor[rgb]{0.81,0.36,0.00}{\textbf{#1}}}
\newcommand{\OtherTok}[1]{\textcolor[rgb]{0.56,0.35,0.01}{#1}}
\newcommand{\PreprocessorTok}[1]{\textcolor[rgb]{0.56,0.35,0.01}{\textit{#1}}}
\newcommand{\RegionMarkerTok}[1]{#1}
\newcommand{\SpecialCharTok}[1]{\textcolor[rgb]{0.00,0.00,0.00}{#1}}
\newcommand{\SpecialStringTok}[1]{\textcolor[rgb]{0.31,0.60,0.02}{#1}}
\newcommand{\StringTok}[1]{\textcolor[rgb]{0.31,0.60,0.02}{#1}}
\newcommand{\VariableTok}[1]{\textcolor[rgb]{0.00,0.00,0.00}{#1}}
\newcommand{\VerbatimStringTok}[1]{\textcolor[rgb]{0.31,0.60,0.02}{#1}}
\newcommand{\WarningTok}[1]{\textcolor[rgb]{0.56,0.35,0.01}{\textbf{\textit{#1}}}}
\usepackage{graphicx,grffile}
\makeatletter
\def\maxwidth{\ifdim\Gin@nat@width>\linewidth\linewidth\else\Gin@nat@width\fi}
\def\maxheight{\ifdim\Gin@nat@height>\textheight\textheight\else\Gin@nat@height\fi}
\makeatother
% Scale images if necessary, so that they will not overflow the page
% margins by default, and it is still possible to overwrite the defaults
% using explicit options in \includegraphics[width, height, ...]{}
\setkeys{Gin}{width=\maxwidth,height=\maxheight,keepaspectratio}
% Set default figure placement to htbp
\makeatletter
\def\fps@figure{htbp}
\makeatother
\setlength{\emergencystretch}{3em} % prevent overfull lines
\providecommand{\tightlist}{%
  \setlength{\itemsep}{0pt}\setlength{\parskip}{0pt}}
\setcounter{secnumdepth}{-\maxdimen} % remove section numbering

\title{Fitting spatial and spatiotemporal models}
\subtitle{FISH 507 -- Applied Time Series Analysis}
\author{Eric Ward}
\date{2 Mar 2021}

\begin{document}
\frame{\titlepage}

\begin{frame}{What we've learned so far}
\protect\hypertarget{what-weve-learned-so-far}{}

\begin{itemize}
\item
  Time series can be useful for identifying structure, improving
  precision, and accuracy of forecasts
\item
  Modeling multivariate time series

  \begin{itemize}
  \tightlist
  \item
    e.g.~MARSS() function, with each observed time series mapped to a
    single discrete state
  \end{itemize}
\item
  Using DFA

  \begin{itemize}
  \tightlist
  \item
    Structure determined by factor loadings
  \end{itemize}
\end{itemize}

\end{frame}

\begin{frame}{Response generally the same variable (not separate
species)}
\protect\hypertarget{response-generally-the-same-variable-not-separate-species}{}

\begin{itemize}
\tightlist
\item
  Making inference about population as a whole involves jointly modeling
  both time series
\end{itemize}

\begin{center}\includegraphics{lec_17_spatiotemporal_files/figure-beamer/unnamed-chunk-1-1} \end{center}

\end{frame}

\begin{frame}{Lots of time series -\textgreater{} complicated covariance
matrices}
\protect\hypertarget{lots-of-time-series---complicated-covariance-matrices}{}

\begin{itemize}
\item
  Last week, in talking about Gaussian process models, we showed the
  number of covariance matrix parameters = \(m*(m+1)/2\)

  \begin{itemize}
  \tightlist
  \item
    problematic for \textgreater{} 5 or so time series
  \end{itemize}
\item
  MARSS solutions: diagonal matrices or `equalvarcov'
\item
  DFA runs into same issues with estimating unconstrained \(R\) matrix
\end{itemize}

\end{frame}

\begin{frame}{Potential problems with both the MARSS and DFA approach}
\protect\hypertarget{potential-problems-with-both-the-marss-and-dfa-approach}{}

\begin{itemize}
\item
  Sites separated by large distances may be grouped together
\item
  Sites close to one another may be found to have very different
  dynamics
\end{itemize}

\end{frame}

\begin{frame}{Are there biological mechanisms that may explain this?}
\protect\hypertarget{are-there-biological-mechanisms-that-may-explain-this}{}

\begin{itemize}
\tightlist
\item
  Puget Sound Chinook salmon

  \begin{itemize}
  \tightlist
  \item
    21 populations generally cluster into 2-3 groups based on genetics
  \item
    Historically large hatchery programs
  \end{itemize}
\item
  Hood canal harbor seals

  \begin{itemize}
  \tightlist
  \item
    Visited by killer whales
  \end{itemize}
\end{itemize}

\begin{center}\includegraphics{lec_17_spatiotemporal_files/figure-beamer/unnamed-chunk-2-1} \end{center}

\end{frame}

\begin{frame}{Motivation of explicitly including spatial structure}
\protect\hypertarget{motivation-of-explicitly-including-spatial-structure}{}

\begin{itemize}
\item
  Adjacent sites can be allowed to covary
\item
  Estimated parameters greatly reduced to 2-5
\end{itemize}

\end{frame}

\begin{frame}{Types of spatial data}
\protect\hypertarget{types-of-spatial-data}{}

Point referenced data (aka geostatistical)

\begin{itemize}
\tightlist
\item
  Typically 2-D, but could be 1-D or 3-D (depth, altitude)
\item
  May be fixed station or random (e.g.~trawl surveys)
\end{itemize}

Point pattern data

\begin{itemize}
\tightlist
\item
  Spatially referenced based on outcomes (e.g.~presence)
\item
  Inference focused on describing clustering (or not)
\end{itemize}

Areal data

\begin{itemize}
\tightlist
\item
  Locations occur in blocks
\item
  counties, management zones, etc.
\end{itemize}

\end{frame}

\begin{frame}{Areal data: Seattle Council Districts}
\protect\hypertarget{areal-data-seattle-council-districts}{}

\includegraphics{DistrictsMap.jpeg}

\end{frame}

\begin{frame}{Computationally convenient approaches}
\protect\hypertarget{computationally-convenient-approaches}{}

CAR (conditionally autoregressive models)

\begin{itemize}
\tightlist
\item
  Better suited for Bayesian methods
\item
  Goal of both is to write the distribution of a single prediction
  analytically in terms of the joint (y1, y2)
\end{itemize}

SAR (simultaneous autregressive models)

\begin{itemize}
\tightlist
\item
  Better suited for ML methods
\item
  Simultaneously model distribution of predicted values
\end{itemize}

`Autoregressive' in the sense of spatial dependency / correlation
between locations

\end{frame}

\begin{frame}{CAR models (Besag 1991)}
\protect\hypertarget{car-models-besag-1991}{}

\[{ Y }_{ i }=\textbf{b}{ X }_{ i }+{ \phi  }_{ i }+{ \varepsilon  }_{ i }\]

\({ X }_{ i }\) are predictors (regression) \({ \phi }_{ i }i\) spatial
component, (aka markov random field) \({ \varepsilon }_{ i }\) residual
error term

\begin{itemize}
\tightlist
\item
  Create spatial adjacency matrix W, based on neighbors, e.g.~
\item
  W(i,j) = 1 if neighbors, 0 otherwise
\item
  W often row-normalized (rows sum to 1)
\item
  Diagonal elements W(i,i) are 0
\end{itemize}

\end{frame}

\begin{frame}{Adjacency matrix example}
\protect\hypertarget{adjacency-matrix-example}{}

\end{frame}

\begin{frame}{CAR models}
\protect\hypertarget{car-models}{}

In matrix form,

\[{ \phi\sim N\left( 0,{ \left( I-\rho W \right)  }^{ -1 }\widetilde { D }  \right) \\ { \widetilde { D }  }_{ ii }={ \sigma  }_{ i } }\]
* Implemented in `spdep', `CARBayes', etc

\end{frame}

\begin{frame}{CAR models}
\protect\hypertarget{car-models-1}{}

\begin{itemize}
\tightlist
\item
  Each element has conditional distribution dependent on others,
\end{itemize}

\(\phi_i \mid \phi_j, \sim \mathrm{N} \left( \sum_{j = 1}^n W_{ij} \phi_j, {\sigma}^2 \right)\)
for \(j \neq i\)

\end{frame}

\begin{frame}{SAR models}
\protect\hypertarget{sar-models}{}

\begin{itemize}
\tightlist
\item
  Simultaneous autoregressive model
\end{itemize}

\[{ \phi \sim N\left( 0,{ \left( I-\rho W \right)  }^{ -1 }\widetilde { D } { \left( I-\rho W \right)  }^{ -1 } \right) \\ { \widetilde { D }  }_{ ii }={ \sigma  }_{ i } }\]

\begin{itemize}
\tightlist
\item
  Remember that the CAR was
\end{itemize}

\[{ \phi \sim N\left( 0,{ \left( I-\rho W \right)  }^{ -1 }\widetilde { D }  \right) \\ { \widetilde { D }  }_{ ii }={ \sigma  }_{ i } }\]

\end{frame}

\begin{frame}[fragile]{Example adjacency matrix}
\protect\hypertarget{example-adjacency-matrix}{}

\begin{itemize}
\item
  \(I-\rho W\)
\item
  Example \(W\) matrix,
\end{itemize}

\begin{verbatim}
##      [,1] [,2] [,3]
## [1,]    0    1    0
## [2,]    1    0    1
## [3,]    0    1    0
\end{verbatim}

\end{frame}

\begin{frame}[fragile]{Example adjacency matrix}
\protect\hypertarget{example-adjacency-matrix-1}{}

\begin{itemize}
\tightlist
\item
  \((I-0.3 W)\), \(\rho=0.3\)
\end{itemize}

\begin{verbatim}
##      [,1] [,2] [,3]
## [1,]  1.0 -0.3  0.0
## [2,] -0.3  1.0 -0.3
## [3,]  0.0 -0.3  1.0
\end{verbatim}

\end{frame}

\begin{frame}[fragile]{Example adjacency matrix}
\protect\hypertarget{example-adjacency-matrix-2}{}

\begin{itemize}
\tightlist
\item
  \((I-0.3 W)\widetilde { D }\)
\item
  Let's assume same variance \textasciitilde{} 0.1
\item
  D = diag(0.1,3)
\end{itemize}

\begin{verbatim}
##       [,1]  [,2]  [,3]
## [1,]  0.10 -0.03  0.00
## [2,] -0.03  0.10 -0.03
## [3,]  0.00 -0.03  0.10
\end{verbatim}

\end{frame}

\begin{frame}{Commonalities of both approaches}
\protect\hypertarget{commonalities-of-both-approaches}{}

\begin{itemize}
\item
  Adjacency matrix W can also instead be modified to include distance
\item
  Models spatial dependency as a function of single parameter \(\rho\)
\item
  Models don't include time dimension in spatial field

  \begin{itemize}
  \tightlist
  \item
    One field estimated for all time steps
  \item
    How could we do this?
  \end{itemize}
\end{itemize}

\end{frame}

\begin{frame}[fragile]{brms() includes both approaches}
\protect\hypertarget{brms-includes-both-approaches}{}

\begin{Shaded}
\begin{Highlighting}[]
\NormalTok{?brms}\OperatorTok{::}\NormalTok{sar}
\NormalTok{?brms}\OperatorTok{::}\NormalTok{car}
\end{Highlighting}
\end{Shaded}

\end{frame}

\begin{frame}{Example: COVID tracking project}
\protect\hypertarget{example-covid-tracking-project}{}

Data from COVID tracking project
\href{https://covidtracking.com/data/download}{link}

\begin{center}\includegraphics{lec_17_spatiotemporal_files/figure-beamer/unnamed-chunk-7-1} \end{center}

\end{frame}

\begin{frame}[fragile]{Differenced change in positive \%}
\protect\hypertarget{differenced-change-in-positive}{}

\begin{itemize}
\tightlist
\item
  Suppose we wanted to model log difference of positive \% cases
\end{itemize}

\begin{Shaded}
\begin{Highlighting}[]
\NormalTok{d =}\StringTok{ }\NormalTok{dplyr}\OperatorTok{::}\KeywordTok{filter}\NormalTok{(d, state }\OperatorTok\StringTok{ }\KeywordTok{rownames}\NormalTok{(W)) }\OperatorTok\StringTok{ }
\StringTok{  }\NormalTok{dplyr}\OperatorTok{::}\KeywordTok{mutate}\NormalTok{( }
    \DataTypeTok{pct =}\NormalTok{ positive}\OperatorTok{/}\NormalTok{totalTestResults) }\OperatorTok\StringTok{ }
\StringTok{  }\NormalTok{dplyr}\OperatorTok{::}\KeywordTok{filter}\NormalTok{(yday }\OperatorTok\StringTok{ }\KeywordTok{c}\NormalTok{(}\DecValTok{122}\NormalTok{,}\DecValTok{366}\NormalTok{)) }\OperatorTok\StringTok{ }
\StringTok{  }\NormalTok{dplyr}\OperatorTok{::}\KeywordTok{group_by}\NormalTok{(state) }\OperatorTok\StringTok{ }
\StringTok{  }\NormalTok{dplyr}\OperatorTok{::}\KeywordTok{arrange}\NormalTok{(yday) }\OperatorTok\StringTok{ }
\StringTok{  }\NormalTok{dplyr}\OperatorTok{::}\KeywordTok{summarize}\NormalTok{(}\DataTypeTok{diff =} \KeywordTok{log}\NormalTok{(pct[}\DecValTok{2}\NormalTok{])}\OperatorTok{-}\KeywordTok{log}\NormalTok{(pct[}\DecValTok{1}\NormalTok{]))}
\end{Highlighting}
\end{Shaded}

\end{frame}

\begin{frame}[fragile]{Differenced change in positive \%}
\protect\hypertarget{differenced-change-in-positive-1}{}

\begin{Shaded}
\begin{Highlighting}[]
\CommentTok{# fit a CAR model}
\NormalTok{fit <-}\StringTok{ }\KeywordTok{brm}\NormalTok{(diff }\OperatorTok{~}\StringTok{ }\DecValTok{1} \OperatorTok{+}\StringTok{ }\KeywordTok{car}\NormalTok{(W), }
           \DataTypeTok{data =}\NormalTok{ d, }\DataTypeTok{data2 =} \KeywordTok{list}\NormalTok{(}\DataTypeTok{W =}\NormalTok{ W), }\DataTypeTok{chains=}\DecValTok{1}\NormalTok{) }
\KeywordTok{summary}\NormalTok{(fit)}
\end{Highlighting}
\end{Shaded}

\end{frame}

\begin{frame}{General challenge with CAR and SAR models}
\protect\hypertarget{general-challenge-with-car-and-sar-models}{}

Wall (2004) ``A close look at the spatial structure implied by the CAR
and SAR models''.

\begin{itemize}
\tightlist
\item
  Note: not a 1-to-1 mapping of distance and correlation
\end{itemize}

\end{frame}

\begin{frame}{Alternative to CAR \& SAR}
\protect\hypertarget{alternative-to-car-sar}{}

\begin{itemize}
\tightlist
\item
  model elements of Q as functions
\item
  Create matrix D, as pairwise distances
\item
  This can be 1-D, or any dimension

  \begin{itemize}
  \tightlist
  \item
    We'll use Euclidian distances for 2-D
  \end{itemize}
\end{itemize}

\end{frame}

\begin{frame}{Squared exponential covariance}
\protect\hypertarget{squared-exponential-covariance}{}

\end{frame}

\begin{frame}{Squared exponential covariance}
\protect\hypertarget{squared-exponential-covariance-1}{}

\end{frame}

\begin{frame}{Which covariance kernel to choose?}
\protect\hypertarget{which-covariance-kernel-to-choose}{}

\begin{center}\includegraphics{lec_17_spatiotemporal_files/figure-beamer/unnamed-chunk-12-1} \end{center}

\end{frame}

\begin{frame}{Which covariance kernel to choose?}
\protect\hypertarget{which-covariance-kernel-to-choose-1}{}

\begin{center}\includegraphics{lec_17_spatiotemporal_files/figure-beamer/unnamed-chunk-13-1} \end{center}

\end{frame}

\begin{frame}{Considerations for time series models}
\protect\hypertarget{considerations-for-time-series-models}{}

Should spatial dependency be included? If so, how to model it?

\begin{itemize}
\tightlist
\item
  Constant
\item
  Time varying
\item
  Autoregressive
\item
  Random walk
\item
  Independent variation
\end{itemize}

\end{frame}

\begin{frame}{Model-based geostatistical approaches}
\protect\hypertarget{model-based-geostatistical-approaches}{}

\begin{enumerate}
\item
  Generalized least squares
\item
  Bayesian methods in spBayes, glmmfields
\item
  INLA models
\item
  Spatial GAMs
\item
  TMB (VAST, sdmTMB)
\end{enumerate}

\end{frame}

\begin{frame}{Method 1: using gls()}
\protect\hypertarget{method-1-using-gls}{}

\begin{itemize}
\tightlist
\item
  Generalized least squares function

  \begin{itemize}
  \tightlist
  \item
    similar syntax to lm, glm, etc.
  \end{itemize}
\item
  Flexible correlation structures

  \begin{itemize}
  \tightlist
  \item
    corExp()
  \item
    corGaus()
  \item
    corLin()
  \item
    corSpher()
  \end{itemize}
\item
  Allows irregularly spaced data / NAs

  \begin{itemize}
  \tightlist
  \item
    unlike Arima(), auto.arima(), etc.
  \end{itemize}
\end{itemize}

\end{frame}

\begin{frame}{WA Snotel sites}
\protect\hypertarget{wa-snotel-sites}{}

\end{frame}

\begin{frame}{Modeling WA Snotel data}
\protect\hypertarget{modeling-wa-snotel-data}{}

We'll use Snow Water Equivalent (SWE) data in Washington state

70 SNOTEL sites

\begin{itemize}
\tightlist
\item
  we'll focus only on Cascades
\end{itemize}

1981-2013

Initially start using just the February SWE data

1518 data points (only 29 missing values!)

\end{frame}

\begin{frame}[fragile]{Use AIC to evaluate different correlation models}
\protect\hypertarget{use-aic-to-evaluate-different-correlation-models}{}

\begin{Shaded}
\begin{Highlighting}[]
\NormalTok{mod.exp =}\StringTok{ }\KeywordTok{gls}\NormalTok{(Feb }\OperatorTok{~}\StringTok{ }\NormalTok{elev, }
  \DataTypeTok{correlation =} \KeywordTok{corExp}\NormalTok{(}\DataTypeTok{form=}\OperatorTok{~}\NormalTok{lat}\OperatorTok{+}\NormalTok{lon,}\DataTypeTok{nugget=}\NormalTok{T), }
  \DataTypeTok{data =}\NormalTok{ y[}\KeywordTok{which}\NormalTok{(}\KeywordTok{is.na}\NormalTok{(y}\OperatorTok{$}\NormalTok{Feb)}\OperatorTok{==}\NormalTok{F }\OperatorTok{&}\StringTok{ }\NormalTok{y}\OperatorTok{$}\NormalTok{Water.Year}\OperatorTok{==}\DecValTok{2013}\NormalTok{),])}
\KeywordTok{AIC}\NormalTok{(mod.exp) =}\StringTok{ }\FloatTok{431.097}

\NormalTok{mod.gaus =}\StringTok{ }\KeywordTok{gls}\NormalTok{(Feb }\OperatorTok{~}\StringTok{ }\NormalTok{elev, }
  \DataTypeTok{correlation =} \KeywordTok{corGaus}\NormalTok{(}\DataTypeTok{form=}\OperatorTok{~}\NormalTok{lat}\OperatorTok{+}\NormalTok{lon,}\DataTypeTok{nugget=}\NormalTok{T), }
  \DataTypeTok{data =}\NormalTok{ y[}\KeywordTok{which}\NormalTok{(}\KeywordTok{is.na}\NormalTok{(y}\OperatorTok{$}\NormalTok{Feb)}\OperatorTok{==}\NormalTok{F }\OperatorTok{&}\StringTok{ }\NormalTok{y}\OperatorTok{$}\NormalTok{Water.Year}\OperatorTok{==}\DecValTok{2013}\NormalTok{),])}
\KeywordTok{AIC}\NormalTok{(mod.gaus) =}\StringTok{ }\FloatTok{433.485}
\end{Highlighting}
\end{Shaded}

\end{frame}

\begin{frame}[fragile]{Diagnostics: fitting variograms}
\protect\hypertarget{diagnostics-fitting-variograms}{}

\begin{Shaded}
\begin{Highlighting}[]
\NormalTok{var.exp <-}\StringTok{ }\KeywordTok{Variogram}\NormalTok{(mod.exp, }\DataTypeTok{form =}\OperatorTok{~}\StringTok{ }\NormalTok{lat}\OperatorTok{+}\NormalTok{lon)}
\KeywordTok{plot}\NormalTok{(var.exp,}\DataTypeTok{main=}\StringTok{"Exponential"}\NormalTok{,}\DataTypeTok{ylim=}\KeywordTok{c}\NormalTok{(}\DecValTok{0}\NormalTok{,}\DecValTok{1}\NormalTok{))}

\NormalTok{var.gaus <-}\StringTok{ }\KeywordTok{Variogram}\NormalTok{(mod.gaus, }\DataTypeTok{form =}\OperatorTok{~}\StringTok{ }\NormalTok{lat}\OperatorTok{+}\NormalTok{lon)}
\KeywordTok{plot}\NormalTok{(var.gaus,}\DataTypeTok{main=}\StringTok{"Gaussian"}\NormalTok{,}\DataTypeTok{ylim=}\KeywordTok{c}\NormalTok{(}\DecValTok{0}\NormalTok{,}\DecValTok{1}\NormalTok{))}
\end{Highlighting}
\end{Shaded}

\end{frame}

\begin{frame}{Exponential variogram}
\protect\hypertarget{exponential-variogram}{}

Semivariance = \(0.5 \quad Var({x}_{1},{x}_{2})\)\\
Semivariance = Sill - \(Cov({x}_{1},{x}_{2})\)

\end{frame}

\begin{frame}{Gaussian variogram}
\protect\hypertarget{gaussian-variogram}{}

\end{frame}

\begin{frame}{Extensions of these models}
\protect\hypertarget{extensions-of-these-models}{}

corExp and corGaus spatial structure useful for wide variety of models /
R packages

Linear/non-linear mixed effect models

\begin{itemize}
\tightlist
\item
  lme() / nlme() in nlme package
\end{itemize}

Generalized linear mixed models

\begin{itemize}
\tightlist
\item
  glmmPQL() in MASS package
\end{itemize}

Generalized additive mixed models

\begin{itemize}
\tightlist
\item
  gamm() in mgcv package
\end{itemize}

\end{frame}

\begin{frame}{Method 2: GAMs}
\protect\hypertarget{method-2-gams}{}

Previous approaches modeled errors (or random effects) as correlated

\begin{itemize}
\tightlist
\item
  Spatial GAMs generally model mean as spatially correlated
\end{itemize}

\end{frame}

\begin{frame}[fragile]{Example with SNOTEL data}
\protect\hypertarget{example-with-snotel-data}{}

First a simple GAM, with latitude and longtide.

\begin{itemize}
\tightlist
\item
  Note we're not transforming to UTM, but probably should
\item
  Note we're not including tensor smooths te(), but probably should
\end{itemize}

\begin{Shaded}
\begin{Highlighting}[]
\NormalTok{d =}\StringTok{ }\NormalTok{dplyr}\OperatorTok{::}\KeywordTok{filter}\NormalTok{(d, }\OperatorTok{!}\KeywordTok{is.na}\NormalTok{(Water.Year), }\OperatorTok{!}\KeywordTok{is.na}\NormalTok{(Feb)) }
\NormalTok{mod =}\StringTok{ }\KeywordTok{gam}\NormalTok{(Feb }\OperatorTok{~}\StringTok{ }\KeywordTok{s}\NormalTok{(Water.Year) }\OperatorTok{+}\StringTok{ }
\StringTok{    }\KeywordTok{s}\NormalTok{(Longitude, Latitude), }\DataTypeTok{data=}\NormalTok{d)}
\NormalTok{d}\OperatorTok{$}\NormalTok{resid =}\StringTok{ }\KeywordTok{resid}\NormalTok{(mod)}
\end{Highlighting}
\end{Shaded}

\end{frame}

\begin{frame}{Ok, let's look at the residuals}
\protect\hypertarget{ok-lets-look-at-the-residuals}{}

First, residuals through time

\begin{center}\includegraphics{lec_17_spatiotemporal_files/figure-beamer/unnamed-chunk-18-1} \end{center}

\end{frame}

\begin{frame}{Now residuals spatially}
\protect\hypertarget{now-residuals-spatially}{}

\begin{itemize}
\tightlist
\item
  We'll use a subset of years here
\end{itemize}

\begin{center}\includegraphics{lec_17_spatiotemporal_files/figure-beamer/unnamed-chunk-19-1} \end{center}

\end{frame}

\begin{frame}[fragile]{Two ways to include spatio-temporal interactions}
\protect\hypertarget{two-ways-to-include-spatio-temporal-interactions}{}

\begin{itemize}
\tightlist
\item
  What do these interactions mean in the context of our SNOTEL data?
\end{itemize}

\begin{enumerate}
\tightlist
\item
  First, spatial field may vary by year
\end{enumerate}

\begin{Shaded}
\begin{Highlighting}[]
\NormalTok{mod2 =}\StringTok{ }\KeywordTok{gam}\NormalTok{(Feb }\OperatorTok{~}\StringTok{ }\KeywordTok{s}\NormalTok{(Water.Year) }\OperatorTok{+}\StringTok{ }
\StringTok{    }\KeywordTok{s}\NormalTok{(Longitude, Latitude, }\DataTypeTok{by=}\KeywordTok{as.factor}\NormalTok{(Water.Year)), }\DataTypeTok{data=}\NormalTok{d)}
\NormalTok{d}\OperatorTok{$}\NormalTok{resid =}\StringTok{ }\KeywordTok{resid}\NormalTok{(mod2)}
\end{Highlighting}
\end{Shaded}

\end{frame}

\begin{frame}[fragile]{Two ways to include spatio-temporal interactions}
\protect\hypertarget{two-ways-to-include-spatio-temporal-interactions-1}{}

\begin{enumerate}
\setcounter{enumi}{1}
\tightlist
\item
  Spatial effect may be continuous and we can include space-time
  interaction
\end{enumerate}

\begin{itemize}
\tightlist
\item
  Why ti() instead of te() ? ti() only is the interaction where te()
  includes the main effects
\end{itemize}

\begin{Shaded}
\begin{Highlighting}[]
\NormalTok{mod2 =}\StringTok{ }\KeywordTok{gam}\NormalTok{(Feb }\OperatorTok{~}\StringTok{ }\KeywordTok{s}\NormalTok{(Longitude, Latitude) }\OperatorTok{+}\StringTok{ }\KeywordTok{s}\NormalTok{(Water.Year) }\OperatorTok{+}\StringTok{ }
\StringTok{    }\KeywordTok{ti}\NormalTok{(Longitude, Latitude,Water.Year, }\DataTypeTok{d=}\KeywordTok{c}\NormalTok{(}\DecValTok{2}\NormalTok{,}\DecValTok{1}\NormalTok{)), }\DataTypeTok{data=}\NormalTok{d)}
\NormalTok{d}\OperatorTok{$}\NormalTok{resid =}\StringTok{ }\KeywordTok{resid}\NormalTok{(mod2)}
\end{Highlighting}
\end{Shaded}

\end{frame}

\begin{frame}{GAM extensions and resources}
\protect\hypertarget{gam-extensions-and-resources}{}

Quickly evolving features for spatio-temporal models

\begin{itemize}
\tightlist
\item
  Random effects
  \href{https://www.rdocumentation.org/packages/mgcv/versions/1.8-34/topics/gamm}{gamm}
\item
  Interface with inla
  \href{https://www.rdocumentation.org/packages/mgcv/versions/1.8-33/topics/ginla}{ginla}
\item
  Bayesian estimation
  \href{https://www.rdocumentation.org/packages/mgcv/versions/1.8-33/topics/bam}{bam}
\end{itemize}

Lots of detailed examples and resources

*e.g.~ESA workshop by Simpson/Pederson/Ross/Miller

\url{https://noamross.github.io/mgcv-esa-workshop/}

\end{frame}

\begin{frame}{Potential limitations of spatial GAMs}
\protect\hypertarget{potential-limitations-of-spatial-gams}{}

\begin{itemize}
\item
  Hard to specify covariance structure explicitly
\item
  Hard to add constraints, like making spatial field be an AR(1) process
\item
  Motivates more custom approaches
\end{itemize}

\end{frame}

\begin{frame}{Bayesian: spBayes}
\protect\hypertarget{bayesian-spbayes}{}

\begin{itemize}
\item
  Package on CRAN,
  \href{https://cran.r-project.org/web/packages/spBayes/index.html}{spBayes}
\item
  Follows general form of CAR / SAR models, in doing spatial regression
\end{itemize}

\[\textbf{ Y }=\textbf{b}\textbf{ X }+{ \omega  }+{ \varepsilon  }\]\\
* \(\omega\) is some spatial field / process of interest\\
* \(\epsilon\) is the residual error

\end{frame}

\begin{frame}{Bayesian: spBayes}
\protect\hypertarget{bayesian-spbayes-1}{}

\begin{itemize}
\item
  spLM / spGLM
\item
  spMvLM / spMvGLM
\item
  spDynLM: space continuous, time discrete
\end{itemize}

\end{frame}

\begin{frame}{Bayesian: spBayes}
\protect\hypertarget{bayesian-spbayes-2}{}

\begin{itemize}
\tightlist
\item
  Slightly more complicated syntax than functions we've been working
  with:
\end{itemize}

Specify:

\begin{itemize}
\tightlist
\item
  Priors on parameters
\item
  Tuning parameters for Metropolis sampling (jumping variance)
\item
  Starting / initial values
\item
  Covariance structure (``exponential'', ``gaussian'', ``matern'', etc)
\item
  Number of MCMC iterations / burn-in, etc.
\end{itemize}

\end{frame}

\begin{frame}[fragile]{Example with SNOTEL data}
\protect\hypertarget{example-with-snotel-data-1}{}

\begin{Shaded}
\begin{Highlighting}[]
\CommentTok{# This syntax is dependent on model parameters. See vignette}
\NormalTok{priors <-}\StringTok{ }\KeywordTok{list}\NormalTok{(}\StringTok{"beta.Norm"}\NormalTok{=}\KeywordTok{list}\NormalTok{(}\KeywordTok{rep}\NormalTok{(}\DecValTok{0}\NormalTok{,p), }
  \KeywordTok{diag}\NormalTok{(}\DecValTok{1000}\NormalTok{,p)), }\StringTok{"phi.Unif"}\NormalTok{=}\KeywordTok{c}\NormalTok{(}\DecValTok{3}\OperatorTok{/}\DecValTok{1}\NormalTok{, }\DecValTok{3}\OperatorTok{/}\FloatTok{0.1}\NormalTok{), }
  \StringTok{"sigma.sq.IG"}\NormalTok{=}\KeywordTok{c}\NormalTok{(}\DecValTok{2}\NormalTok{, }\DecValTok{2}\NormalTok{), }\StringTok{"tau.sq.IG"}\NormalTok{=}\KeywordTok{c}\NormalTok{(}\DecValTok{2}\NormalTok{, }\FloatTok{0.1}\NormalTok{))}

\CommentTok{# Phi is spatial scale parameter, sigma.sq is spatial variance, }
\CommentTok{# tau.sq = residual}
\NormalTok{starting <-}\StringTok{ }\KeywordTok{list}\NormalTok{(}\StringTok{"phi"}\NormalTok{=}\DecValTok{3}\OperatorTok{/}\FloatTok{0.5}\NormalTok{, }\StringTok{"sigma.sq"}\NormalTok{=}\DecValTok{50}\NormalTok{, }\StringTok{"tau.sq"}\NormalTok{=}\DecValTok{1}\NormalTok{)}

\CommentTok{# variance of normal proposals for Metropolis algorithm}
\NormalTok{tuning <-}\StringTok{ }\KeywordTok{list}\NormalTok{(}\StringTok{"phi"}\NormalTok{=}\FloatTok{0.1}\NormalTok{, }\StringTok{"sigma.sq"}\NormalTok{=}\FloatTok{0.1}\NormalTok{, }\StringTok{"tau.sq"}\NormalTok{=}\FloatTok{0.1}\NormalTok{)}

\NormalTok{m}\FloatTok{.1}\NormalTok{ <-}\StringTok{ }\KeywordTok{spLM}\NormalTok{(y}\OperatorTok{~}\NormalTok{X}\DecValTok{-1}\NormalTok{, }\DataTypeTok{coords=}\NormalTok{cords, }
  \DataTypeTok{n.samples=}\DecValTok{10000}\NormalTok{, }
  \DataTypeTok{cov.model =} \StringTok{"exponential"}\NormalTok{, }\DataTypeTok{priors=}\NormalTok{priors, }
  \DataTypeTok{tuning=}\NormalTok{tuning, }\DataTypeTok{starting=}\NormalTok{starting)}
\end{Highlighting}
\end{Shaded}

\end{frame}

\begin{frame}[fragile]{Coefficients need to be extracted}
\protect\hypertarget{coefficients-need-to-be-extracted}{}

\begin{Shaded}
\begin{Highlighting}[]
\CommentTok{##recover beta and spatial random effects}
\NormalTok{burn.in <-}\StringTok{ }\DecValTok{5000}
\NormalTok{m}\FloatTok{.1}\NormalTok{ <-}\StringTok{ }\KeywordTok{spRecover}\NormalTok{(m}\FloatTok{.1}\NormalTok{, }\DataTypeTok{start=}\NormalTok{burn.in, }\DataTypeTok{verbose=}\OtherTok{FALSE}\NormalTok{)}
\end{Highlighting}
\end{Shaded}

\end{frame}

\begin{frame}{Standard MCMC diagnostics}
\protect\hypertarget{standard-mcmc-diagnostics}{}

Output is of class `mcmc'

\end{frame}

\begin{frame}{Lots of recent applications of CAR/SAR / spBayes}
\protect\hypertarget{lots-of-recent-applications-of-carsar-spbayes}{}

\href{https://cdnsciencepub.com/doi/abs/10.1139/cjfas-2017-0481}{Nelson
et al.~2018}

\href{https://mc-stan.org/users/documentation/case-studies/mbjoseph-CARStan.html}{Joseph
2016}

\href{https://academic.oup.com/condor/advance-article/doi/10.1093/ornithapp/duaa065/6053195}{Smith
\& Edwards 2020}

\href{https://www.sciencedirect.com/science/article/abs/pii/S1364815219310412}{Finley
\& Banerjee 2020}

\end{frame}

\end{document}
